\section{Definition von Skills für Anwendung} \label{Anwendungsskills}
	Um die benötigten Skills für die Anwendung zu definieren, werden im ersten Schritt die allgemeinen Arbeitsschritte (Tab. \ref{tab:Arbeitsschritte}) basierend auf dem mechanischen Aufbau (siehe Kapitel \ref{Mechanischer_Aufbau}) festgelegt. Dabei wird von der Ausgangssituation ausgegangen, dass alle Teile in ihren Lagerpositionen abgelegt sind, der Roboter sich in der Home-Position befindet und alle Komponenten eingeschaltet sowie betriebsbereit sind.
	
	\begin{table}[ht]
		\centering
		\colorlet{BFH-table}{BFH-MediumBlue!10}
		\colorlet{BFH-tablehead}{BFH-MediumBlue!50}
		\begin{bfhTabular}{lll}
			Schritt: 	& Beschreibung:				&Komponente:								
			\\\hline
			1			& Position von Platte 1 in Lagerung erkennen							& K
			\\\hline
			2			& Platte 1 mittels L-Stück ausrichten									& R, G, K
			\\\hline
			3 			& Position von Platte 2 in Lagerung erkennen							& K
			\\\hline
			4			& Platte 2 und  Platte 1 zusammenführen									& R, G, K
			\\\hline
			5			& Position der Befestigungslöcher in den Platten ermitteln				& K
			\\\hline
			6			& Position von Befestigungsblech in Lagerung erkennen		 			& K	
			\\\hline
			7			& Befestigungsblech an Montageposition bringen 							& R, G, K
			\\\hline
			8			& Position von Stift 1 in Lagerung erkennen								& K
			\\\hline
			9			& Stift 1 an korrekte Position bringen									& R, G
			\\\hline
			10			& Befestigungsblech mit Platte 1 verbinden (mit Stift) 					& R, G
			\\\hline
			11			& Wiederholen von Schritt 8 – 10 für Stift 2 bis 3						&
		\end{bfhTabular}
		\begin{tablenotes}
			\small
			\item (K = Kamerasystem | R = Roboter | G = Greifer)
		\end{tablenotes}
		\caption{Arbeitsschritte}
		\label{tab:Arbeitsschritte}
	\end{table}
	
	Eine detaillierte Auflistung der Arbeitsschritte wird im Anhang beigefügt. Aus diesem lässt sich erkennen, dass sich diverse Schritte mit kleinen Anpassungen wiederholen.  Diese sich wiederholenden Arbeitsschritte definieren die Skills. Ein entscheidender Aspekt dabei ist, dass der Roboter und der Kraftsensor als separate Komponenten betrachtet werden. Der Kraftsensor erweitert die Fähigkeiten des Roboters zwar und damit dessen Skills, jedoch kann der Roboter auch ohne Kraftsensor betrieben werden. Der Kraftsensor wird als eigene Objektklasse abgebildet, jedoch besitzt dieser keinen eigenen Skill. Folgende Skills wurden definiert, welche den Prozess abdecken:
	
	\begin{table}[ht]
		\centering
		\colorlet{BFH-table}{BFH-MediumBlue!10}
		\colorlet{BFH-tablehead}{BFH-MediumBlue!50}
		\begin{bfhTabular}{lll}
			Komponenten: 	& Skill:								& Bemerkung:								
			\\\hline
			Kamerasystem			& - Bild aufnehmen				& 
			\\\hline
									& - Objekt erkennen				& 
			\\\hline
									& - Greiferposition ermitteln	& 
			\\\hline
			Roboter					& - Position anfahren			& 
			\\\hline
									& - Kontrolliert bewegen		& Bei Kraftüberschreitung wird gestoppt
			\\\hline
			Greifer (mit Sensor)	& - Backenposition anfahren		& 									
		\end{bfhTabular}
		\begin{tablenotes}
			\small
			\item (Kamerasystem = Kamera + Vision)
		\end{tablenotes}
		\caption{Definierte Skills für Anwendung}
		\label{tab:Definierte_Skills}
	\end{table}
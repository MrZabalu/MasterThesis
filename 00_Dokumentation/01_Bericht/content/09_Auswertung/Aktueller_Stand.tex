\section{Aktueller Stand und Erkenntnisse} \label{Aktueller_Stand}
	Der mechanische Aufbau für die Umsetzung der definierten Anwendung konnte umgesetzt in Betrieb genommen werden. Der einfach gehaltene Aufbau erfüllt dabei seinen Zweck und konnte für das Testen der Software eingesetzt werden. Die Software erfüllt dabei die grundlegend gestellten Anforderungen, befindet sich jedoch noch in der Entwicklung und somit im Prototypenstatus. 
	\\
	Mit dem Prototypen kann bereits gezeigt werden, dass die definierte Struktur funktioniert. Abläufe können erstellt und durchgeführt werden. Dabei greifen diese auf die definierten Skills zu, welche wiederum mit den umgesetzten Objekten interagieren. Der allgemeine Workflow der Software konnte dadurch getestet und in ersten Schritten optimiert werden. Die umgesetzten Objekte erfüllen dabei ihre spezifischen Aufgaben. 
	\vspace{2mm}
	Dies sind einige der Vorteile der umgesetzten Software-Struktur:
	\begin{itemize}
		\item Einfache Umsetzung von Abläufen.
		\item Einfache Anpassung von bestehenden Abläufen.
		\item Übersichtlich- und Verständlichkeit der Software. 
		\item Gesamte Funktionalität der Anlage wird in der SPS zusammengefasst.
		\item Anlagen- und Prozessmodell können separat voneinander aufgebaut werden.
		\item Komponenten können einfach ergänzt oder ausgetauscht werden, ohne grosse Veränderungen am Prozessmodell.
	\end{itemize} 
	\vspace{3mm}
	
	Einige vorgesehene Elemente der Software konnten jedoch noch nicht umgesetzt werden. Die Einbindung des Vision-Systems in den Prozess wurde aus zeitgründen weggelassen. Auch die Umsetzung des HMI für die Anlagenkomponenten und des Gesamtsystem-HMI wurde noch nicht gemacht. Das momentane HMI ist nur für das schnelle Testen vorgesehen. Das System kann dabei eingeschaltet und der Prozess gestartet werden.
	\\
	Auch der komplette Ablauf konnte zum Stand der Dokumentation noch nicht umgesetzt werden. Der Fokus während der Entwicklung wurde auf eine stabile Struktur gelegt, da dies das Fundament der Software darstellt. Viel Zeit wurde dadurch in das Testen und optimieren der Struktur investiert. Sobald dieser robust und sauber umgesetzt ist, ist die Umsetzung des kompletten Ablaufs kein grosser Aufwand mehr. Der Ablauf wurde entsprechend soweit aufgebaut, dass alle Funktionalitäten und Interaktionen getestet werden konnten. 
	\\
	Ein aktueller Fehler, welcher noch behoben werden muss, tritt beim Starten von Skills auf (in Kapitel \ref{Prozessmodell_Sequenzaufbau} beschrieben). Dies und weitere kleiner Optimierungen sollten in den Wochen, nach Abgabe dieser Arbeit, noch umgesetzt werden können. 
	
	\newpage
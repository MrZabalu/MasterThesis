\section{Weiterführende Arbeiten} \label{Weiterführende_Arbeiten}

	\textbf{Short-Term:}
	\vspace{2mm}  
	\\
	Die unmittelbar weiterführenden Arbeiten beziehen sich auf das Abschliessen der Software für die definierte Anwendung. Dazu gehören folgende Punkte:
	
	\begin{itemize}
		\item Allgemeine Optimierung der Software (Stabilität, Bug-Behebung)
		\item Umsetzung von HMI und sammeln von Erfahrungen mit TE2000
		\item Implementierung von Kamerasystem in Prozess
	\end{itemize}
	
	\vspace{3mm}  
	
	\textbf{Mid-Term:}
	\vspace{2mm}  
	\\
	In Kontakt treten mit Unternehmen, für welche eine solche Software-Struktur interessant sein könnte, um folgende Fragen zu klären: 
	
	\begin{itemize}
		\item Was ist die allgemeine Meinung zu dieser Struktur und den Vorteilen, welche diese mit sich bringt
		\item Für welche Unternehmen könne eine solche Struktur interessant sein
		\item Wie könnte diese Struktur bei Unternehmen implementiert werden
		\item Welche Anforderungen stellen Unternehmen an die Struktur
	\end{itemize}
	
	\vspace{3mm}  
	
	\textbf{Long-Term:}
	\vspace{2mm}  
	\\
	Langfristig muss die Struktur weiter ausgebaut und optimiert werden. Das Ziel ist, dass diese alle Anforderungen abdeckt, welche die Industrie an eine solche Software stellt. Dafür muss sich weiter mit den Möglichkeiten von Beckhoff und TwinCAT auseinander gesetzt werden. 
	\\
	Eine weiteres Ziel ist die Entwicklung eines separaten Software-Tools (basierend auf z.B. \verb|C#|), mit welchen ein TwinCAT-Programm auf Basis der definierten Struktur konfiguriert werden kann. Mit dem "TwinCAT Automation Interface" können TwinCAT-Programme automatisch erzeugt und konfiguriert werden. Es könnte möglich sein, dass in einem separaten Programm nur  die Skills und Objekte konfiguriert werden müssen und anschliessend wird die komplette Struktur in TwinCAT automatisch, anhand der Konfiguration, erstellt.
	\\
	Die Möglichkeiten dieser Schnittstelle müssten jedoch in einem ersten Schritt getestet und verstanden werden. Jedoch könnte ein solches Tool interessant für Unternehmen sein, da die Software-Entwicklungszeit massiv reduziert und vereinfacht werden könnte.
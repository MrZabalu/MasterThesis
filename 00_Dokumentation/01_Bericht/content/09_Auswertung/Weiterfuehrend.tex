\section{Weiterführende Arbeiten} \label{Weiterführende_Arbeiten}

	\textbf{Kurzfristig:}
	\vspace{2mm}  
	\\
	Die unmittelbar weiterführenden Arbeiten beziehen sich auf das Abschliessen der Software für die definierte Anwendung. Dazu gehören folgende Punkte:
	
	\begin{itemize}
		\item Allgemeine Optimierung der Software (Stabilität, Bug-Behebung).
		\item Umsetzung von HMI und sammeln von Erfahrungen mit TE2000.
		\item Implementierung von Kamerasystem in Prozess.
	\end{itemize}
	\vspace{3mm}  

	Mit dem gesammelten Know-How und den gemachten Erfahrungen sollte die gesamte Struktur in einer weiteren Iteration nochmals optimiert und angepasst werden. Viele Elemente können einfacher, effizienter oder eleganter gelöst werden. 
	
	\vspace{3mm}  
	
	\textbf{Mittelfristig:}
	\vspace{2mm}  
	\\
	In Kontakt treten mit Unternehmen, für welche eine solche Software-Struktur interessant sein könnte, um folgende Fragen zu klären: 
	
	\begin{itemize}
		\item Was ist die allgemeine Meinung zu dieser Struktur und ihrer Vorteile.
		\item Für welche Unternehmen könne eine solche Struktur interessant sein.
		\item Wie könnte diese Struktur bei Unternehmen implementiert werden.
		\item Welche Anforderungen stellen Unternehmen an die Struktur.
	\end{itemize}
	\vspace{3mm}  

	Das Ziel sollte sein, einen Überblick über den Markt und seiner Interessen und Anforderungen zu haben. Daraus kann eingeschätzt werden, welches Potential eine solche Software-Struktur im Markt haben könnte. 
	
	\vspace{3mm}  
	
	\textbf{Langfristig:}
	\vspace{2mm}  
	\\
	Langfristig sollen die Erkenntnisse aus der Erarbeitung der Software-Struktur mit der Analyse des Marktes verbunden werden. Die Struktur soll anhand der Bedürfnisse des Marktes ausgebaut und optimiert werden. Mit dem Ziel, möglichst alle Anforderungen der Industrie (innerhalb eines sinnvollen Kontextes) mit der Software abdecken zu können. Dafür muss das Know-How in Bezug auf Beckhoff und TwinCAT weiter ausgebaut und vertieft werden. 
	\\
	Dazu gehört die Entwicklung eines separaten Software-Tools (basierend auf z.B. \verb|C#|), mit welchen ein TwinCAT-Programm auf Basis der definierten Struktur konfiguriert werden kann. Mit dem «TwinCAT Automation Interface» können TwinCAT-Programme automatisiert erzeugt und konfiguriert werden. Somit könnte es möglich sein, über ein solches Programm Skills und Objekte zu konfigurieren welches anschliessend aufgrund dieser Konfiguration ein komplettes TwinCAT-Projekt erstellt, basierend auf der entwickelten Struktur.
	\\
	Die Möglichkeiten dieser Schnittstelle müssten jedoch in einem ersten Schritt getestet und verstanden werden. Jedoch könnte ein solches Tool interessant für Unternehmen sein, da die Software-Entwicklungszeit massiv reduziert und vereinfacht werden könnte.
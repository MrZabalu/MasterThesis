Das Prozessmodell ist für den korrekten Ablauf des Prozesses zuständig, Arbeitsplan, Sequenzen und Skills sind Teil des Prozessmodells. Wie in Kapitel \ref{Grundlagenfragen für Thematik} beschrieben, wird das Prozessmodell so definiert, dass dieses anlagenunabhängig aufgebaut werden kann. Das  Modell muss nicht wissen, welche Komponenten im System vorhanden sind, sondern nur die Fähigkeiten dieser. Dadurch können Prozess- und Anlagenmodell parallel entwickelt werden, was die Kosten und Entwicklungszeit verringern kann. Zudem erlaubt die klare Trennung eine flexible Anpassung oder den Austausch von Komponenten im Anlagenmodell, ohne das Prozessmodell zu beeinflussen. Umgekehrt können neue Prozesse schnell und unkompliziert implementiert werden, ohne dass detaillierte Kenntnisse der Anlagenkomponenten erforderlich sind.
\\
Diese Flexibilität führt insgesamt zu einer effizienteren und schnelleren Entwicklung sowie Inbetriebnahme der Software.

Das Hauptziel des Projekts, eine Roboteranwendung einfach und standardisiert programmieren zu können, wurde aus meiner Sicht erfolgreich erreicht. In TwinCAT wurde eine Struktur entwickelt, die eine übersichtliche und unkomplizierte Erstellung von Prozessen ermöglicht. Sobald Objektklassen und Skills definiert sind, lässt sich der Prozess lediglich durch das Erstellen von Schrittabläufen realisieren – hierfür sind keine tiefgehenden Programmierkenntnisse erforderlich. Das Entwickeln der Objektklassen setzt jedoch ein gutes Verständnis der zu implementierenden Komponenten voraus.
\\
Das Projekt hat auch gezeigt, dass solche Anwendungen komplett innerhalb einer SPS umgesetzt werden können. Die SPS und vor allem TwinCAT, bieten viele Möglichkeiten eine solche Software zu realisieren und unterschiedlichste Komponenten in das System zu integrieren. 
\\
Die Software wurde erfolgreich mit einem Versuchsaufbau getestet, wobei noch Optimierungspotenziale identifiziert werden konnten. Das Testen stellte einen essentiellen Bestandteil der iterativen Entwicklung der Software-Struktur dar und trug massgeblich zur Verfeinerung des Systems bei.
\\
Für mich war das Projekt eine äusserst spannende und lehrreiche Erfahrung, geprägt von vielen Höhen, aber auch einigen Herausforderungen. Zu Beginn war ich aufgrund des offenen Projektauftrags skeptisch und unsicher, in welche Richtung sich das Vorhaben entwickeln würde. Mit der Zeit wurden jedoch der Rahmen und die Zielrichtung immer klarer und somit auch interessanter.
\\
Im Verlauf des Projekts verlagerte sich der Fokus von der reinen Skill-Entwicklung hin zur Gestaltung einer allgemeinen Software-Struktur, die für die Implementierung und Nutzung dieser Skills erforderlich ist. Die Entwicklung dieser Struktur war für mich zweifellos ein Höhepunkt der Arbeit und bereitete mir grosse Freude. Besonders motivierend war das positive Feedback aus der Industrie zu meinem Ansatz, das meinen Antrieb zur Umsetzung nochmals deutlich gesteigert hat.
\\
Ein weiterer spannender Aspekt war die intensive Arbeit mit TwinCAT. Für die Umsetzung der Software musste und konnte viel Neues gelernt werden, was jedoch auch diverse Herausforderungen mit sich brachte. Besonders zeitaufwendig war die Entwicklung der Objektklassen, die eine gründliche Einarbeitung erforderte. Dies hatte einen erheblichen Einfluss auf den Zeitplan. So wurden beispielsweise allein für die Integration einer Modbus-RTU-Verbindung in TwinCAT drei Wochen benötigt, um diese erfolgreich umzusetzen.
\\
Dieser Prozess umfasste mehrere Schritte: Zunächst musste die Arbeitsweise und Struktur des Kommunikationsprotokolls verstanden werden. Anschliessend galt es, dessen Einsatz in TwinCAT zu analysieren und herauszufinden, warum es anfänglich nicht funktionierte. Es wurden mögliche Lösungen erarbeitet, und letztlich konnte eine alternative Schnittstelle erfolgreich in TwinCAT integriert werden. Dieser langwierige Prozess war zwar herausfordernd, doch die Freude war umso grösser, als die Kommunikation schliesslich einwandfrei funktionierte. Dies ist nur eines von vielen Beispielen für die Herausforderungen, die bei der Entwicklung der Objektklassen gemeistert werden mussten.
\\
Abschliessend kann ich sagen, dass ich mit meiner Arbeit sehr zufrieden und stolz bin. Es ist für mich klar, dass ich diese Thematik auch nach Abschluss meines Masters weiterverfolgen möchte. Ich bin gespannt, wie sich das Projekt in Zukunft weiterentwickeln wird.

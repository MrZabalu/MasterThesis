\newglossaryentry{SFC}{
	name=SFC,
	description={\textbf{Sequential Function Chart:} Eine grafische Programmiersprache nach IEC 61131-3, die in TwinCAT zur Modellierung und Steuerung sequenzieller Abläufe genutzt wird. Sie besteht aus Schritten, Übergängen und Aktionen.}
}

\newglossaryentry{CFC}{
	name=CFC,
	description={\textbf{Continuous Funtcion Chart:} Grafische Programmiersprache nach IEC 61131-3 in TwinCAT, die eine flexible, frei platzierbare Darstellung von Steuerungslogik ermöglicht. Ideal für komplexe Signalflüsse.}
}

\newglossaryentry{ST}{
	name=ST,
	description={\textbf{Structured Text:} Eine textbasierte Programmiersprache nach IEC 61131-3 in TwinCAT, die eine strukturierte Syntax für komplexe Steuerungslogik bietet. Ideal für mathematische und datenbasierte Operationen.}
}

\newglossaryentry{SPS}{
	name=SPS,
	description={\textbf{Speicherprogrammierbare Stuerung:} Ein elektronisches Steuerungssystem zur Automatisierung von Maschinen und Prozessen. Es basiert auf einem Mikroprozessor und wird mit Programmiersprachen nach IEC 61131-3, wie in TwinCAT, programmiert.}
}

\newglossaryentry{EGM}{
	name=EGM,
	description={\textbf{Externally Guided Motion:} Ein Steuerungsmodus von ABB-Robotern, der eine externe Echtzeitführung der Roboterbewegungen ermöglicht. Dabei werden Positions- und Geschwindigkeitsbefehle direkt von einem externen System an den Robotercontroller übermittelt, was hochpräzise und dynamische Bewegungen erlaubt.}
}

\newglossaryentry{ROS}{
	name=ROS,
	description={\textbf{Robot Operating System:} Ein flexibles Framework zur Entwicklung von Robotersoftware. Es bietet Tools und Bibliotheken für Funktionen wie Sensorintegration, Steuerung und Bewegungsplanung. ROS ermöglicht modulare, plattformübergreifende Robotikanwendungen und wird in Forschung und Industrie weltweit eingesetzt.}
}

\newglossaryentry{MTP}{
	name=MTP,
	description={\textbf{Module Type Package:} Ein Standard zur modularen Prozessautomatisierung, der eine herstellerunabhängige Integration von Automatisierungskomponenten ermöglicht. MTP definiert eine standardisierte Schnittstelle für die Kommunikation und Interaktion zwischen Modulen und einem Prozessleitsystem, was Flexibilität und Skalierbarkeit fördert.}
}

\newglossaryentry{HMI}{
	name=HMI,
	description={\textbf{Human Machine Interface:} Eine Benutzerschnittstelle, die die Interaktion zwischen Mensch und Maschine ermöglicht. HMIs visualisieren Daten, steuern Prozesse und erleichtern die Überwachung und Bedienung automatisierter Systeme.}
}

\newglossaryentry{VDI}{
	name=VDI,
	description={\textbf{Verein Deutscher Ingenieure:} Ein deutscher Ingenieurverein, der technische Normen, Richtlinien und Empfehlungen entwickelt. VDI ist auch bekannt für seine Veröffentlichungen, Weiterbildungen und Fachveranstaltungen in verschiedenen Ingenieurdisziplinen.}
}

\newglossaryentry{TCP}{
	name=TCP,
	description={\textbf{Tool Center Point:} Der Tool Center Point ist der Referenzpunkt am Roboterwerkzeug, an dem alle Bewegungen und Positionierungen des Roboters ausgerichtet sind.}
}






\section{Gate-Plan} \label{Gate-Plan}

	Der Gate-Plan besteht aus 4 Phasen. Jede Phase wird durch ein Gate begonnen und abgeschlossen. Innerhalb der Phasen werden die verschiedenen Arbeitspakete erarbeitet. 

	\begin{figure}[h!]
		\centering
		\includegraphics[width=1\textwidth]{03_Vorgehen_fuer_die_Thesis/Gateplan}
		\captionsetup{justification=centering}
		\caption{Definierter Gate-Plan}
		\label{fig:Gateplan}
	\end{figure}
	
	\textbf{Phase 1:} \vspace{2mm} 
	\\
		\begin{tabularx}{\textwidth}{@{}>{}p{7em} X@{}}
			Beschreibung: & 
			Phase 1 beschäftigt sich mit den Grundlagen der Hard- und Software. Die relevanten Komponenten müssen definiert und die notwendigen Fragen bezüglich der Software-Struktur geklärt werden.
			\\
			
			Output: & 
			\begin{itemize}
				\item Alle Komponenten für den Versuchsaufbau sind bekannt
				\item Die allgemeine Software-Struktur wurde definiert
			\end{itemize}
		\end{tabularx}
	
	\textbf{Phase 2:} \vspace{2mm} 
	\\
		\begin{tabularx}{\textwidth}{@{}>{}p{7em} X@{}}
			Beschreibung: & 
			Innerhalb von Phase 2 wird der Hardware-Teil abgeschlossen. Dafür wird der Versuchsaufbau konstruiert und gebaut. Zusätzlich werden die Software-Schnittstellen der Komponenten analysiert und definiert. 
			\\
			
			Output: & 
			\begin{itemize}
				\item Gebauter Versuchsbau für das Testen der Software
				\item Es ist bekannt, wie alle Komponenten in die Software implementiert werden können
			\end{itemize}
		\end{tabularx}
	
	\textbf{Phase 3:} \vspace{2mm} 
	\\
		\begin{tabularx}{\textwidth}{@{}>{}p{7em} X@{}}
			Beschreibung: & 
			Phase 3 stellt die konkrete Umsetzung des skill-basierten Ansatzes in der Software dar. Hierbei wird iterativ die Software entwickelt. Die umfasst die Umsetzung der Struktur, wie auch die Entwicklung der Skills und Objektklassen.  
			\\
	
			Output: & 
			\begin{itemize}
				\item Funktionaler Prototyp, welcher an der Versuchsanlage getestet werden kann
			\end{itemize}
		\end{tabularx}
	
	\textbf{Phase 4:} \vspace{2mm} 
	\\
		\begin{tabularx}{\textwidth}{@{}>{}p{7em} X@{}}
			Beschreibung: & 
			Die letzte Phase beschäftigt sich mit der Entwicklung eines HMI.  
			\\
			
			Output: & 
			\begin{itemize}
				\item Über HMI bedienbarer Prototyp, welcher an der Versuchsanlage getestet werden kann
			\end{itemize}
		\end{tabularx}
	
	